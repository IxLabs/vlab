\chapter{vLab Overview}
\label{chapter:Chapter 3}

This chapter aims to provide a broad view of the \project\ application in terms of how the user can interact with it, what are the main use cases and how it works in general.
The focus is on who can use \project\ and what can one do with the application, namely what are the main use cases for it.
Then, a short description about what actually happens in the back-end is made for the user to better understand the workflow.

\section{Who Uses \project?}
\label{sec:who-uses-vlab}

Basically anyone can use \project, given the fact that it is an open source project and it's freely available.
Even though this is true, not everybody will end up needing this tool because it is specialized for tasks related to networking and kernel modules testing in general.
A developer needs to use his time as efficiently as possible, thus he needs to automate every job that he can.
It is pretty difficult to check whether the module under test is full of bugs or not without the possibility to automate things, especially when developing kernel modules.

Taking these facts into consideration, there are several use cases for \project\ discussed next.

\abbrev{VM}{Virtual Machine}
\subsection{Turning On or Off the VMs}
\label{sub-sec:turning-on-off-vms}

\fig[scale=1]{src/img/diagrams/usecase-start-stop.pdf}{img:usecase-start-stop}{Starting/Stopping VMs Usecase Diagram}

This scenario presents the user who normally needs to start or stop the Virtual Machines that he needs for testing.
It is exactly what \labelindexref{Figure}{img:usecase-start-stop} expresses, having the developer centered on either trying to turn the environment on and then doing some other tasks like testing or running debugging command and then stopping it when he's done.
He can either start or stop every Virtual Machine, which in \project\ are referred to as Hosts, or he can choose which one to turn on or off individually, in separate commands.
This first case would be the preferred one for automated testing, while the latter would only be used for debugging purposes, let's say one would need to start only some machines from a given network topology and then run debugging commands like \texttt{tcpdump} to check only that part in the topology.

Of course, the application is designed such that the user cannot accidentally start the Virtual Machines if they are already started, or in the process to be turned on, as well as for the process of turning them off.
This behavior not only easier to implement and test, but it is also the best choice in this case, as it guarantees that the hosts corresponding to the topology can be started once and only once, which in fact means the number of states in which the application can be at some point is deterministic.

\subsection{Spawning an Xterm Session on Each VM}
\label{sub-sec:spawn-xterm}

\fig[scale=1]{src/img/diagrams/usecase-spawn-xterm.pdf}{img:usecase-spawn-xterm}{Spawning a New Xterm Session}

Another use case is presented in \labelindexref{Figure}{img:usecase-spawn-xterm}, which describe how the user can spawn Xterm sessions for each machine.
This is a handy feature when the user needs to quickly get access to some, or all VMs at once.
Basically once starting an Xterm session on a VM, the user has full access to that machine, such that he can easily view its state through different commands available on any Linux based operating system.

Also, the user has the possibility to open several Xterm sessions on the same VM, such that he can do some tests that involve using two or more terminals to be opened at a time.

\subsection{Running Commands on VMs}
\label{sub-sec:run-cmd}

\fig[scale=1]{src/img/diagrams/usecase-run-cmd.pdf}{img:usecase-run-cmd}{Running a command on VMs}

Besides spawning Xterm sessions to control the Hosts, the user also has the option to send commands directly through \project\ CLI.
This use case, described in \labelindexref{Figure}{img:usecase-run-cmd}, offers another simple way for quick sending commands to Hosts, without the need to actually connect to them via SSH or Xterm sessions.
The feature is similar to the way Mininet handles the sending of commands to hosts, being able to write something like this:

\lstset{label=lst:ping-example}
\begin{lstlisting}
vLab> host1 ping host2
\end{lstlisting}

This creates a connection to the machine named \texttt{host1} and runs the command \texttt{ping host2} in the context of the host1 machine, printing the output of the command on \project\ CLI.
It basically redirects standard error and standard output to \project\ CLI such that the user can see whether he issued a wrong command, or there was an error when running the actual command.

\section{Typical Workflow in \project}
\label{sec:typical-workflow}

This section presents how actions are linked one to another in \project.
They are shown graphically in \labelindexref{Figure}{img:activity-overview} for a better understanding.

\fig[scale=0.8]{src/img/diagrams/activity-overview.pdf}{img:activity-overview}{Activity diagram overview}

The entry point of the activity diagram is when the user actually starts the application.
Further on, the application waits for commands from the user.
This basically replaces the current shell prompt where \project\ is run with the internal CLI defined within the application.
This CLI is responsible for waiting for any input from the user, and then passing the command to the rest of the application.
The fastest way to get to the final point is to enter the command for terminating the application, which has a couple aliases described in \labelindexref{Chapter}{chapter:Chapter 4}.

The other main paths through the application are represented by the commands for starting and stopping the Virtual Machines.
There is a command that can start every host at once, or another which starts only one host given its hostname.
While the machines are in the process to be started, the user cannot issue any other new commands, as they would need to be turned on anyway for being able to receive the commands.
Before actually trying to start a machine, the application verifies whether it was already turned on or not.
The counterpart of the start command is of course, the stop one, which does the reverse.

Any other command is simply executed and then the output is printed to the console where the CLI is run.
These kind of instructions represent the part where the user can begin running tests, or do whatever he wants with any of the machines that he just spawned.

\section{Hosts and Topologies}
\label{sec:hosts-and-topologies}

One of the main features of \project\ is the ability to create different network topologies in a simple manner, through configuration files.
It is designed such that one can create an entire laboratory with virtual computers linked together in different ways in just a matter of minutes.
This is achieved thanks to the simple manner in which the configuration files are constructed.

\subsection{Generic Network Topologies}
\label{sub-sec:generic-network-topologies}

These configuration files for the network topology can be created either manually, starting from an existing example, or they can be generated from an open source visual tool named MiniEdit\footnote{\url{http://techandtrains.com/category/miniedit/}}.
It was originally created for Mininet, about which we discussed in \labelindexref{Section}{sec:mininet}.
\project\ was designed with compatibility in mind, thus the configuration files for topologies are mostly identical with those generated from MiniEdit, making it possible for the user to generate them in a more user friendly manner, from a Graphical User Interface tool.

\fig[scale=0.8]{src/img/mini-edit-topology.pdf}{img:mini-edit-topology}{Topology generated with MiniEdit}

In \labelindexref{Figure}{img:mini-edit-topology} MiniEdit can be seen in action while creating a small topology, with two Hosts connected directly to a Switch.
This can be achieved by simply dragging the components on the drawing pane, and then exporting the configuration to a JSON\footnote{\url{http://json.org/}} file.
The file generated for this particular topology can be found entirely in the Appendix, \labelindexref{Listing}{lst:config-one-switch-two-hosts}.

As this project aims to give the users the opportunity to spend their time effectively, making it possible to generate topologies from an external tool was a must have feature, as they could have a better view of what network they are designing for their testing purpose.
Although this offers a great advantage, it is not as flexible as the manual configuration which this project also supports.

\subsection{Host Specific Configurations}
\label{sub-sec:host-specific-configurations}

Besides the topologies, this application also needs another type of configuration file, which specifies various parameters, like where to load the Linux Kernel image from, how much RAM should be allocated for each Virtual Machine that represents a host and others.
A more detailed description of the VM specific configurations is given in \labelindexref{Section}{sec:arh-overview} which defines the architecture of this application.

This type of configurations is \project\ specific, so currently they can only be edited manually, as there is no visual tool implemented yet to allow for user-friendly configurations.
Even so, the file format is the same as for topologies, namely JSON, which is a simple enough format that should be easily understood by a developer that is trying to test network modules and therefore has some knowledge of how various configuration formats work.

As a short preview of what happens when a new Host is added to a topology in this application, one can imagine a box that is a real computer, which has several input/output ports and some network card with which it communicates to the rest of the world.
A Virtual Machine performs exactly this set of interactions, but in a way that it simulates the real hardware in software.
This is why we can have multiple instances of an operating system being run on the same physical machine at the same time.
Continuing with the box analogy, the computer that hosts the VMs uses a virtualization software that actually creates these boxes, allocates the necessary resources and manages them accordingly.
From the perspective of the user of this application, all this is transparent.
The user only knows about hosts, which can be turned on, ran for any amount of time with any kind of tests, and then turned off.

These hosts are automatically linked to their neighbors within the topology at startup, and if the network level configurations are done properly and if their drivers are working well, then they should be able to communicate with each other.
After having a working configuration, then the developer could begin to change the modules with the ones that he is developing and begin testing them.
This should basically be all that the user needs to do for setting up the environment and running his tests.
