\chapter{Conclusion and Future Work}
\label{chapter:Chapter 6}

This chapter will summarize what was achieved so far with \project\ and what features I will work on in the future.

\section{Conclusion}
\label{sec:conclusion}

\project\ is an efficient tool for automating the development and testing of kernel network modules.
It does this in a flexible manner, allowing users to easily configure the topologies they want to test, as well as the properties of the nodes they simulate.

Even though in its current state \project\ does not support all kinds of \texttt{Nodes}, we reached our goals by having an application that works not only for small network topologies, but also on bigger networks with more \texttt{Nodes}.
It was indeed a challenge to make it flexible and design it to be easily integrated with other software, but in the end it proved very rewarding as we worked with multiple Virtual Machines on the same physical host.

Also, it is very pleasant to see that after we made the evaluation of \project, it proved to actually scale well and to add a significant improvement for the developers by reducing the time they spend with setting up their testing environment.

\section{Future Work}
\label{sec:future-work}

One important aspect of this project is the fact that it is available open source, free to use by anybody.
It can be found on GitHub in as a Git repository\footnote{\url{https://github.com/IxLabs/vlab}}, where people can always get the latest source code, as well as contributing to the project by reporting bugs, updating the code for their needs, and even submitting patches with new features.

As possible future features that can be implemented we enumerate a few of them:
\begin{itemize}
  \item Integration with Mininet
  \item Possibility to create links between \texttt{Switches}
  \item Adding another \texttt{Node} implementation, the \texttt{Controller} for \texttt{Switches}
\end{itemize}

The first feature is plausible, as both Mininet and \project\ work with network topologies, but in fact \project\ goes in a different direction when spawning VMs.
Nevertheless, the architecture was designed to be flexible enough, having this possibility in mind from the beginning.
The other two features are also good candidates, as they would add even more values for people testing \texttt{Switches}.
