\chapter{Introduction}
\label{chapter:intro}

\todo{Short description on what is the problem and why do we need vLab}

\section{How is a protocol tested?}
\label{sec:proto-testing}

\todo{Short description on how a protocol can be tested}

\subsection{Buying hardware}
\label{sub-sec:proto-testing-hardware}

\todo{It's expensive, not scalable, takes too long, etc}

\subsection{Create virtual machines using existing solutions}
\label{sub-sec:proto-testing-vms}

\todo{Present existing solutions like VMWare, VirtualBox etc...
It's scalable, but expensive; Configuration takes too much time and it requires good
hardware}

\subsection{Virtualization at process level (Mininet)}
\label{sub-sec:proto-testing-mininet}

\todo{* It uses processes for simulating nodes (switches and hosts)
* Kernel modules cannot be tested}

\section{Project Description}
\label{sec:proj-desc}

\subsection{Introducing vLab, a testing solution based on Qemu and custom configs}
\label{sub-sec:proj-desc-intro}

\subsection{Objective}
\label{sub-sec:proj-desc-objective}

\todo{Objective: Ne propunem să creem un sistem minimal care să simuleze o rețea cu 
număr configurabil de host-uri și switch-uri;
(dorim sa obtinem o aplicate care sa ne permita sa simulam usor si rapid o retea) -
adaugat lista de feature-uri (vLab CLI, ssh, seriala, xterm, config-uri usor de
scris)}

\subsection{What is presented next}
\label{sub-sec:proj-desc-following}

\todo{Short description of what is presented in the next chapters
}
